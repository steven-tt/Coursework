\documentclass[11pt,letterpaper]{beamer}
\usetheme{Berkeley}
\usepackage[ascii]{inputenc}
\usepackage{amsmath}
\usepackage{amsfonts}
\usepackage{amssymb}
\usepackage{graphicx}
\usepackage{booktabs}
%\usepackage[left=1.00in, right=1.00in, top=1.00in, bottom=1.00in]{geometry}

\author{Steven Turner}
\title{Population Modeling}
\date{2016}

\begin{document}
	\frame{\titlepage}
\section{Introduction}
	\begin{frame}
		\frametitle{Introduction}
		\begin{itemize}
			\item exponential, logistic, Gompertz 
			\item Populations: US and Munich, Germany
			\item Analyzing Models
			\begin{itemize}
				\item The models ability to predict
				\item How close the data fits the model
			\end{itemize}
		\end{itemize}
	\end{frame}
	
\section{The Models}

	\begin{frame}
	\frametitle{PPGR and Differential Equations}
	Per Unit Population Growth Rate (PPGR)$ =\frac{1}{p}\frac{dp}{dt} $\\
	\begin{table}
		\resizebox{\textwidth}{!}{
		\begin{tabular}{|c|c|c|c|c|}
			\hline Model & Differential Equation & PPGR & $ \lim\limits_{p\rightarrow 0}\text{PPGR} $ & $ \lim\limits_{p\rightarrow L}\text{PPGR} $ \\[1ex]
			exponential & $ \frac{dp}{dt}=rp $ & $ r $ & $ r $ & $ r $\\[1ex]
			logistic & $ \frac{dp}{dt}=rp\left(1-\frac{p}{L}\right) $ & $ r\left(1-\frac{p}{L}\right) $ & $ r $ & $ 0 $\\[1ex]
			Gompertz & $ \frac{dp}{dt}=rp\log\left(\frac{L}{p}\right) $ & $ r\log\left(\frac{L}{p}\right) $ & $ \infty $ & $ 0 $\\[1ex] \hline
		\end{tabular}}
	\end{table}
	\end{frame}


\section{The Project}

	\begin{frame}
	\frametitle{The Project}
	\begin{columns}
		\begin{column}{.5\textwidth}
			\begin{table}
				\resizebox{\textwidth}{!}{
				\begin{tabular}{|c|c|}
					\hline \multicolumn{2}{|c|}{Solutions to differential equations}\\[1ex]
					\hline Model & Solutions\\[1ex]
					exponential & $ P(t)=Ae^{rt} $\\[1ex]
					logistic & $ P(t)=\frac{L}{1+Ae^{-rt}} $\\[1ex]
					Gompertz & $ P(t)Le^{-be^{-rt}} $\\[1ex] \hline
				\end{tabular}}
			\end{table}
		\end{column}
		\begin{column}{.5\textwidth}
			The project steps
			\begin{itemize}
				\item Data for US and Munich, Germany
				\item Used Python to estimate the parameters in the solutions
				\item Found $ \text{R}^2 $: This is our measure of fit
				\item Found \% error for the actual last data point and the estimated point using the Leave One Out Method
			\end{itemize}
		\end{column}
	\end{columns}

	\end{frame}
	
\section{Results}

\subsection{US}

	\begin{frame}
		\frametitle{Testing the Model}
		\begin{columns}
			\begin{column}{.5\textwidth}
				\begin{table}
					\resizebox{\textwidth}{!}{
					\begin{tabular}{|c|c|c|}
						\hline \multicolumn{3}{|c|}{Leave One Out Method}\\[1ex]
						\hline & $ \text{R}^2 $ error & \% error\\[1ex]
						\hline exponential & 0.9837 & 10.58\% \\[1ex]
						\hline logistic & 0.9971 &  -3.81\%\\[1ex]
						\hline Gompertz & 0.9989 &  -1.28\% \\[1ex]
						\hline
					\end{tabular}}
				\end{table}
			\end{column}
			\begin{column}{.5\textwidth}
				\includegraphics[width=\textwidth]{USPop_Prediction}
			\end{column}
		\end{columns}
	\end{frame}

\subsection{Munich}
	\begin{frame}
		\frametitle{Testing the Model}
		\begin{columns}
			\begin{column}{.5\textwidth}
				\begin{table}
					\resizebox{\textwidth}{!}{
					\begin{tabular}{|c|c|c|}
						\hline \multicolumn{3}{|c|}{Leave One Out Method}\\[1ex]
						\hline & $ \text{R}^2 $ error & \% error\\[1ex]
						\hline exponential & 0.9383  & 7.56\% \\[1ex]
						\hline logistic & 0.9909  & -2.83\% \\[1ex] 
						\hline Gompertz & 0.9898  & -1.26\% \\[1ex] 
						\hline
					\end{tabular}}
				\end{table}
				
			\end{column}
			\begin{column}{.5\textwidth}
				\includegraphics[width=\textwidth]{GPop_Prediction}
			\end{column}
		\end{columns}
	\end{frame}
	

	
\section{Conclusion}

	\begin{frame}
		\frametitle{Conclusion}
		\begin{itemize}
			\item Due to the Gompertz model, for both populations, having the lowest percent error and the largest $ \text{R}^2 $ error we decided it was the better model.\\
			\item Realistically none are very good:\\
			\begin{itemize}
				\item Immigration and emigration\\
				\item Always increasing
			\end{itemize}
		\end{itemize}
		
		% 
	\end{frame}
\end{document}